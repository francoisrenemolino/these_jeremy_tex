\chapter*{}


\section*{Résumé}
L’imagerie par résonnance magnétique (IRM) permet maintenant d’observer différents types
de tissus avec des résolutions de plus en plus fines. L’arbre vasculaire artériel et veineux est explorable
et les flux peuvent y être caractérisés de façon non invasive. Le versant artériel de l’arbre vasculaire
peut être obtenu par une imagerie dite par « temps de vol » et le versant veineux par une imagerie en
contraste de phase. Le développement de reconstructions de cartographies de susceptibilité
magnétique (QSM) permet d’améliorer le niveau de détails atteignable sur les veines en fournissant
en plus la possibilité de quantifier des paramètres physiologiques comme la saturation veineuse en
oxygène. La mise en place d’algorithmes et outils dédiés permet la reconstruction in-silico d’une
architecture cohérente sujet-spécifique. Par ailleurs grâce à l’emploi de séquences de la dynamique
telles que le contraste de phase dynamique et l’imagerie par marquage des protons artériels du sang,
les débits artériels, veineux, et perfusionnels sont mesurables. L’intégralité de ces acquisitions est non
invasive, donc applicable à l’ensemble des sujets passant des IRM. Sur la base de ces données
anatomiques et dynamiques, un modèle complet et sujet-spécifique de l’hydrodynamique
intracrânienne est proposé. Le flux sanguin et cérébro-spinal est décrit dans ce modèle par les
équations bilans fondamentales de l’hydrodynamique : conservation de la masse, conservation de la
quantité de mouvement. Pour tenir compte de l’adaptation du diamètre des vaisseaux aux pressions
on introduit pour chaque compartiment un paramètre d’élasticité de la paroi et une équation
correspondante. Grâce aux données d’imagerie IRM, les compartiments sanguins des artères aux
veines, le parenchyme cérébral et le système ventriculaire sont inclus. Le modèle permet de simuler la
répartition des flux et des pressions dans les différents compartiments de la vascularisation du patient
ainsi que d’évaluer les effets d’occlusions localisées sur l’ensemble de l’architecture.\\

{\bf Mots clefs: biomarqueurs, IRM multimodale, modélisation, hydrodynamique intracrânienne, QSM,
ASL}

\section*{Abstract}
The magnetic resonance imaging (MRI) allows the observation of various kind of tissues with
always increasing resolution. The arterial and venous vascular trees can be explored, and the flows can
be characterized in a noninvasive way. As an example, the arterial part of the tree can be obtained
using so-called “Time Of Flight” MRI, and the venous part with phase contrast techniques. The
development of quantitative susceptibility maps (QSM) improves the level of details achievable
regarding veins; furthermore, it provides a new way to estimate physiological parameters such as
venous saturation in oxygen. Eventually the implementation of dedicated algorithms and tools allows
the in-silico reconstruction of a subject-specific coherent architecture. Moreover, due to the use of
dynamic imaging sequences such as the dynamic phase contrast imaging and the arterial spin labeling,
the arterial, venous and cerebral blood flow are measurable. All of these sequences are noninvasive
and so usable on every subjects. Based on these anatomical and dynamics data, a full subject-specific
model of the brain hydrodynamics is proposed here. The blood and cerebrospinal flow are described
using basic balance equations of the hydrodynamics: continuity and momentum. To take into account
of the adaptation of vessel diameter to the pressure, a wall elasticity parameter is added for each
compartment together with the corresponding equation. Thanks to the MRI data, all the blood
compartments, from arteries to vein, the cerebral parenchyma and the ventricular system are
included. The model is able to simulate the flow and pressure repartition in all compartments of the
subjects as well as show the impact of a located occlusion on the whole architecture.\\
{\bf Keywords: biomarkers, multimodal MRI, modeling, intracranial hydrodynamics, QSM, ASL}

